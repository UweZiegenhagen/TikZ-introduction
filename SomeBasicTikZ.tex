%!TEX TS-program = pdflatex
\documentclass[12pt,english]{beamer}

\usepackage[utf8]{inputenc}
\usepackage[T1]{fontenc}
\usepackage{babel}
\usepackage{graphicx}
\usepackage{csquotes}
\usepackage{xcolor}
\usepackage{listings}



\definecolor{hellgelb}{rgb}{1,1,0.8}
\definecolor{colKeys}{rgb}{0,0,1}
\definecolor{colIdentifier}{rgb}{0,0,0}
\definecolor{colComments}{rgb}{1,0,0}
\definecolor{colString}{rgb}{0,0.5,0}

\lstset{%
    float=hbp,%
    language=[LaTeX]TeX,
    basicstyle=\ttfamily\small, %
    identifierstyle=\color{colIdentifier}, %
    keywordstyle=\color{colKeys}, %
    stringstyle=\color{colString}, %
    commentstyle=\color{colComments}, %
    columns=flexible, %
    tabsize=2, %
    frame=single, %
    extendedchars=true, %
    showspaces=false, %
    showstringspaces=false, %
    numbers=left, %
    numberstyle=\tiny, %
    breaklines=true, %
    backgroundcolor=\color{hellgelb}, %
    breakautoindent=true, %
    captionpos=b%,
    morekeywords={draw}
}

\lstset{literate=%
    {Ö}{{\"O}}1
    {Ä}{{\"A}}1
    {Ü}{{\"U}}1
    {ß}{{\ss}}1
    {ü}{{\"u}}1
    {ä}{{\"a}}1
    {ö}{{\"o}}1
    {~}{{\textasciitilde}}1
}


\usepackage{tikz}

\IfFileExists{plex-sans.sty}{%
\usepackage[sfdefault]{plex-sans}%
}{
% no plex sans
}

\usetheme[progressbar=frametitle]{metropolis}           % Use metropolis theme
 
\makeatletter
\setlength{\metropolis@titleseparator@linewidth}{1pt}
\setlength{\metropolis@progressonsectionpage@linewidth}{1pt}
\setlength{\metropolis@progressinheadfoot@linewidth}{1pt}
\makeatother

\title{Some very basic TikZ}
\author{Dr. Uwe Ziegenhagen}
\institute{www.uweziegenhagen.de}

\begin{document}

\begin{frame}

\maketitle

\end{frame}

\begin{frame}

\tableofcontents

\end{frame}

\section{History}

\begin{frame}
\frametitle{History}

\begin{itemize}
	\item TikZ and PGF were originally developed by Till Tantau together with the Beamer package
	\item  \TeX\ packages for creating graphics programmatically 
	\item Aim: create sophisticated graphics in a more or less intuitive way 
\end{itemize}

\end{frame}

\begin{frame}[fragile]
\frametitle{Command and Environment}

\begin{itemize}
	\item \verb|\usepackage{tikz}|
	\item \texttt{tikzpicture} environment
	\begin{itemize}
		\item \verb|\begin{tikzpicture}|
		\item \verb|\end{tikzpicture}|
	\end{itemize}
	\item \verb|\tikz| command
\end{itemize}

\end{frame}

\begin{frame}[fragile]
\frametitle{A Simple Line}

\begin{lstlisting}
\begin{tikzpicture}
	\draw (0,0) -- (1,1)--(2,0);
\end{tikzpicture}
\end{lstlisting}

\begin{tikzpicture}
	\draw (0,0) -- (1,1)--(2,0);
\end{tikzpicture}

Let's add some grid!
\end{frame}

\begin{frame}[fragile]
\frametitle{A Simple Line}

\begin{lstlisting}
\begin{tikzpicture}
	\draw[step=1.0,gray,thin] (-1,-1) grid (6,4);
	\draw (0,0) -- (1,1)--(2,0);
\end{tikzpicture}
\end{lstlisting}

\begin{tikzpicture}
	\draw[step=1.0,green,thin] (-1,-1) grid (6,4);
	\draw (0,0) -- (1,1)--(2,0);
\end{tikzpicture}

Let's add some grid!
\end{frame}



\end{document}